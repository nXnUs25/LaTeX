Wraz ze wzrostem zapotrzebowania na pozyskanie nowych klient�w, a tym samym kontrakt�w oraz z naciskami ze strony organizacji ochrony �rodowiska, aby zredukowa� liczb� drukowanych dokument�w, coraz wi�cej firm decyduje si� na generowanie i wymian� dokument�w mi�dzy pracownikami (i/lub swoimi oddzia�ami) w formie elektronicznej. Je�li wymiana tych dokument�w dokonywana jest w w�skim gronie pracownik�w, mo�emy by� w 100\% pewni, �e dokument jest autentyczny i w formie niezmienionej. Problem zaczyna si�, gdy musimy wymienia� dokumenty z naszymi pracownikami lub partnerami handlowymi w r�nych miastach lub nawet krajach. Wtedy nie mo�emy mie� pewno�ci, �e dokument wys�any jest tym samym dokumentem, kt�ry otrzymali�my. Przy dzisiejszej technologii dokument taki m�g� zosta� przechwycony przez osoby niepowo�ane (trzecie) i/lub zmieniony w celu zmylenia adresata lub korzy�ci maj�tkowych.

Rozwi�zaniem problemu stwierdzenia autentyczno�ci dokumentu oraz pewno�ci, �e nadawca jest tym, za kogo si� podaje, jest podpis elektroniczny (cyfrowy). Narz�dziem, kt�re nam to w prosty spos�b umo�liwia jest j�zyk XML, a dok�adniej \textit{XML Signature}.
