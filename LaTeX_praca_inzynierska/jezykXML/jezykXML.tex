To jest rozdzia� o XML.

\section{XML Signature}
To sekcja o XML Signature\cite{xml_signature_syntax}.

\subsection{Struktura dokumentu}
Tu jest opis struktury elementu XML Signature.

\scriptsize\begin{verbatim}
<Signature ID?> 
    <SignedInfo>
        <CanonicalizationMethod/>
        <SignatureMethod/>
        (<Reference URI? >
            (<Transforms>)?
            <DigestMethod>
            <DigestValue>
        </Reference>)+
    </SignedInfo>
    <SignatureValue> 
   (<KeyInfo>)?
   (<Object ID?>)*
</Signature>
\end{verbatim}

\normalsize Prosty przyk�ad zastosowania.

\scriptsize\begin{verbatim}
<Signature Id="MyFirstSignature" xmlns="http://www.w3.org/2000/09/xmldsig#"> 
   <SignedInfo> 
      <CanonicalizationMethod Algorithm="http://www.w3.org/2006/12/xml-c14n11"/> 
      <SignatureMethod Algorithm="http://www.w3.org/2000/09/xmldsig#dsa-sha1"/> 
      <Reference URI="http://www.w3.org/TR/2000/REC-xhtml1-20000126/"> 
         <Transforms> 
            <Transform Algorithm="http://www.w3.org/2006/12/xml-c14n11"/> 
         </Transforms> 
         <DigestMethod Algorithm="http://www.w3.org/2000/09/xmldsig#sha1"/> 
         <DigestValue>dGhpcyBpcyBub3QgYSBzaWduYXR1cmUK.../DigestValue> 
      </Reference> 
   </SignedInfo> 
   <SignatureValue>...</SignatureValue> 
   <KeyInfo> 
      <KeyValue>
         <DSAKeyValue> 
            <P>...</P><Q>...</Q><G>...</G><Y>...</Y> 
         </DSAKeyValue> 
      </KeyValue> 
   </KeyInfo> 
</Signature>
\end{verbatim}
\normalsize

\subsubsection{Element \textit{SignedInfo}}

\subsubsection{Element \textit{SignatureValue}}

\subsubsection{Element \textit{KeyInfo}}

\subsubsection{Element \textit{Object}}

\subsection{Elementy dodatkowe}

\subsubsection{Element \textit{Manifest}}

\subsubsection{Element \textit{SignaturePropeties}}
