\hspace{1cm} Celem pracy jest wykonanie projektu, a następnie stworzenie strony wspomagającej zdalne nauczanie. \\
\ \\
Projekt będzie zawierał krótki kurs instruujący w jaki sposób można zainstalować i postawić platformę do zdalnego nauczania. W kursie tym zostaną omówione krok po kroku czynności jakie należy wykonać aby móc korzystać z przywilejów jakie daje nam e-learning. Zostaną omówione zasady instalacji i konfiguracji platformy. Większość decyzji podejmowanych podczas pierwszej fazy instalacji i konfiguracji ma wpływ na późniejsze doświadczenia użytkowników~-~w tym nie tylko uczniów i nauczycieli, ale również administratorów witryny. Następnym krokiem będzie tworzenie struktury witryny. Do struktury witryny należeć będą takie czynności jak tworzenie kategorii kursów, jak i samych kursów. Kolejny krok pokaże jak dodawać podstawowy materiał do kursu. Gdzie w większości kursów online podstawowy materiał składa się ze stron internetowych, które są wyświetlane przez uczniów. Oczywiście będą też zaprezentowane inne rodzaje statycznego materiału edukacyjnego np. takie jak strony tekstowe, odnośniki do innych kursów, etykiety i katalogi plików. \\
\ \\
Stworzenie całego systemu e-learning-owego będzie finalnym zadaniem. Strona internetowa będzie musiała umożliwiać przejście z każdej części kursu do dowolnego miejsca na witrynie. Witryna będzie korzystała z usług jakie daje reCAPTCHA, oraz z narzędzia wspomagającego pracę administratorów Geolocation. \\
\ \\
CAPTCHA to program, który pomaga stwierdzić, czy nasz użytkownik jest człowiekiem, czy też to komputer. Prawdopodobnie każdy widział ich - kolorowe obrazy z zniekształconym tekstem na dole Web formularzy rejestracyjnych. CAPTCHA jest wykorzystywana przez wiele stron internetowych w celu zapobiegania nadużyciom z "botów", czyli zautomatyzowanych programów zwykle pisanych do generowania spamu. Żaden program komputerowy nie potrafi czytać zniekształconych tekstów, ale ludzie mogą, więc boty nie są w stanie przejść przez tereny chronione CAPTCHA. \\
\ \\
GeoIP oferuje przedsiębiorstwom nieinwazyjny sposób na określenie geograficznego położenia oraz innych informacji na temat ich użytkowników w czasie rzeczywistym. Kiedy osoba odwiedza stronę internetową, GeoIP może określić kraj, region, miasto, kod pocztowy i numer kierunkowy gości odwiedzających. Ponadto GeoIP może dostarczyć informacji, takich jak długość / szerokość geograficzną, szybkość połączenia, ISP firmy, nazwy domeny, czy też adres IP używa anonimowego proxy. \\
\ \\
Do wykonania witryny wspomagającej proces zdalnego nauczania posłużę się jednym z LMS\footnote{LMS~-~Learning~Management~System (System~zarządzania~nauczaniem)} Moodle. Każdy LMS praktykuje swoje podejście, który kształtuje doświadczenia użytkowników i zachęca do konkretnego sposobu użytkowania. Środowisko takie może zachęcać do systematycznego nauczania poprzez udostępnianie zasobów w odpowiedniej kolejności i utrzymują porządek w każdym kursie. \\
Znaczenie nazwy platformy Moodla pozwala na zrozumienie jej podejścia do nauczania poprzez internet. \\
\ \\
"Słowo Moodle jest akronimem utworzonym od nazwy \textbf{Modular Object-Oriented Dynamic Learning Environment} (modułowe, dynamiczne, zorientowane obiektowo środowisko nauczania), co jest użyteczne przede wszystkim dla programistów i teoretyków nauczania. Słowo to jest także czasownikiem opisującym proces leniwego, od niechcenia, dochodzenia do poznania czegoś, robienia rzeczy w sposób, jaki uważa się za słuszny, przyjemnego majstrowania, które często sprzyja inwencji i wnikliwości. Odnosi się to zarówno do sposobu, w jaki Moodle się rozwinął, jak i do sposobu, w jaki uczeń lub nauczyciel uczą się lub nauczają w kursie online. Każdy kto używa Moodle, jest \textbf{moodlerem} ." \cite{dokumentacja_moodle} \\
\ \\
Platforma Moodle'a umożliwia uczniom i nauczycielom naukę online, którą można realizować poprzez: \\
	\begin{itemize}
		\item strony internetowe, które mogą być przeglądane w dowolnej kolejności,
		\item kursy z pokojami rozmów przeprowadzanych pomiędzy uczniami a nauczycielami,
		\item fora dyskusyjne, na których użytkownicy mogą oceniać wiadomości pod względem ich adekwatności i wnikliwości,
		\item warsztaty online pozwalające uczniom oceniać i recenzować pracę innych uczniów,
		\item zaimportowanie ankiety umożliwiające nauczycielowi oceniać opinie uczniów o postępie kursu,
		\item katalogi przeznaczone dla nauczycieli pozwalające udostępniać uczniom pliki.
	\end{itemize}
\ \\
Zgodnie z tą ideą tworzone kursy są elementami włączającymi uczestników szkoleń w własny rozwój poprzez samokształcenie i tworzenie społeczności uczącej się wzajemnie od siebie, dzielącej się własnymi doświadczeniami, opiniami. Ogólna zasada nauczania mówi, że ludzie uczą się najlepiej, gdy wchodzą w interakcję z materiałem. Lekcje przeprowadzone w Moodle'u zawierające różne elementy nauczania, zmuszają uczestnika do interakcji z tymi materiałami, co znacznie różni taki system od tradycyjnego prowadzenia lekcji. Różnica ta jest analogiczna do różnicy między wykładem a dyskusją. \\
