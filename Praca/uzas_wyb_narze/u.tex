\hspace{1cm} Przy wyborze narzędzia do tworzenia witryny wspomagającej zdalne nauczanie brałem pod uwagę następujące czynniki, które teraz wymieniem i pokrótce opisze:
	\begin{itemize}
		\item popularność
		\item koszt wykonania witryny
		\item społeczność
		\item skutecznością w realizowaniu celów dydaktycznych
	\end{itemize} 
\section{Popularność} \label{roz:popularnosc_moodle}
\hspace{1cm}  Obecnie najbardziej popularną platformą open source jest platforma Moodle'a. Platforma jest najbardziej znana i najczęściej instalowana do nauczania przez Internet. Jest ona prosta w obsłudze, bogata w funkcje. Zbudowana jest na solidnej filozofii edukacyjnej i ma ogromną społeczność, która wspiera i rozwija platformę. Może śmiało konkurować z wielkimi komercyjnymi serwisami w zakresie zestawów funkcji i jest łatwy do rozszerzenia. Stąd tak duża popularność tej że platformy.
\section{Koszt wykonania witryny} \label{roz:koszt}
\hspace{1cm} Koszt postawienia nie wielkiej platformy Moodle'a jest równy zeru. Projekt jest jak już wcześniej pisałem open source. Dla osób które wcześniej nie spotkały się z z tego typu sformułowaniem jest trudne do zrozumienia jak potężna jest ta idea, i jak zmieniła świat programistów. Pomysł sam w sobie jest prosty: \textit{open source} oznacza, że użytkownicy mają dostęp do źródła kodów platformy. Każdy użytkownik może zaglądać do kodów i modyfikować je na swój własny sposób. Więc czemu jest to tak ważne ? Każdy może pobrać i używać Moodle za darmo, użytkownicy mogą tworzyć nowe moduły, funkcje, naprawiać bugi, poprawiać wydajność, lub po prostu uczyć się jak inni ludzie rozwiązują programistyczne problemy. Moodle może zostać zainstalowany na każdym serwerze obsługującym PHP i umożliwia użycie darmowych baz danych (MySQL, PostgreSQL itp.). Prawie każdy serwer spełnia te warunki.
\section{Społeczność} \label{roz:spolecznosc}
Moodle posiada bardzo dużą i aktywną społeczność, są to osoby które używają Moodle'a jak i zarówno osoby, które tworzą nowe funkcje i udoskonalenia. Każdy ma dostęp do tej społeczności pod adresem \href{http://moodle.org/}{moodle.org} jak również zapisać się na kurs korzystania z Moodle. Można tam znaleźć ludzi którzy są skłoni pomóc nowym użytkownikom w postawieniu i uruchomieniu platformy, w rozwiązywaniu problemów i przy efektywnym używaniu Moodle. Na obecną chwile na \href{http://moodle.org/}{moodle.org} jest zarejestrowanych 300,000 użytkowników i ponad 30,000 stron w 195 krajach. Ta społeczność ma również wkład w przetłumaczeniu platformy na ponad 70 języków. Moodle zawdzięcza tak duże sukcesy właśnie dzięki tak dużej społeczności, gdzie zawsze znajdzie się ktoś kto ma odpowiedz na pytanie lub ktoś kto służy radą. 
\section{Filozofia edukacji} \label{roz:fil_edu} 
\hspace{1cm} Większość CMS-ów była budowana wokół narzędzi, nie pedagogiki. Zarówno większość komercyjnych CMS-ów była również budowana mając na uwadze same narzędzia, gdzie Moodle zostało zaprojektowane i budowane z myślą skutecznego uczenia. Konstruktywizm społeczny jest oparty na założeniu, że ludzie uczą się najlepiej, gdy są zaangażowani w społecznym procesie konstruowania wiedzy poprzez akt budowy materiałów dla innych. Określenie proces społeczny oznacza, że uczenie się jest to coś, co robimy w grupach. \\
Można stwierdzić, że kształcenie na odległość, realizowane w systemie Moodle jest technologią pedagogiczną. Poniżej przedstawiona zostanie tabela\ref{tab:pedagogika}~\ref{tab:cd_pedagogika}~\ref{tab:cd1_pedagogika} (źródło:\cite{pedagogika}) ilustruje potwierdzenie tego założenia.
	\begin{table}[!h]
		\centering
		\caption[Analiza technologii pedagogicznej w Moodle'u]{Analiza technologii pedagogicznej kształcenia na odległość na podstawie wykorzystania systemu MOODLE}
		\label{tab:pedagogika}
		\begin{tabular}{|p{4cm}|p{4cm}|p{9cm}|} \hline
			 \textbf{Charakterystyka technologii pedagogicznej} & \textbf{Opis} & \textbf{Opis technologii pedagogicznej
kształcenia na odległość na podstawie
wykorzystania systemu MOODLE} \\ \hline
			identyfikacja & & \\ \hline
			nazwa technologii & nazwa technologii & technologia pedagogiczna kształcenia na odległość na podstawie wykorzystania systemu MOODLE \\ \hline
			konceptualna część (opis idei, hipotez, zasad technologii) & docelowe założenia i orientacje; podstawowe idee i zasady; pozycja uczącego się w procesie kształcenia & wspomaganie nowych stylów nauki, przede wszystkim kognitywnego, kreatywnego i konstruktywistycznego [Piaget, Papert, Juszczyk, Kwiecicki, Le Blank, Dewey, Bruner, Wygotski], w tym ułatwienie różnych form komunikacji i grupowych form nauki, wzajemnej oceny, kierowania uczniami, także możliwość prostej zamiany ról: uczeń – nauczyciel – twórca (materiałów dydaktycznych) kursów dystansowych. Uczący się ma możliwość samodzielnego konstruowania swojej nauki (czas, miejsce, tempo, treści merytoryczne, tematy projektów), jest aktywna strona procesu nauczania. \\ \hline
		właściwości zawartości kształcenia & orientacja na osobowe struktury (WUN – wiedza, umiejętności, nawyki);\newline objętość i charakter kształcenia, dydaktyczna struktura planu szkolnego, materiału, programów, formy przedstawienia & orientacja na struktury osobowe (WUN), jak również na rozwój umiejętności samodzielnej nauki;\newline objętość i charakter kształcenia zależą od celów nauki, adresata, charakteru przedmiotowego obszaru, przygotowania uczniów itd.;\newline dydaktyczna struktura szkolnego planu, materiału, programów – modułowa;\newline forma przedstawienia – multimedialna, hipertekstowa. Uczący się posiada ciągły dostęp do wszystkich zasobów edukacyjnych \\ \hline
		\end{tabular}
	\end{table}

	\begin{table}[!h]
		\centering
		\caption[c.d. Analiza technologii pedagogicznej w Moodle'u]{Analiza technologii pedagogicznej kształcenia na odległość na podstawie wykorzystania systemu MOODLE}
		\label{tab:cd_pedagogika}
		\begin{tabular}{|p{4cm}|p{4cm}|p{9cm}|} \hline
			\textbf{Charakterystyka technologii pedagogicznej} & \textbf{Opis} & \textbf{Opis technologii pedagogicznej
kształcenia na odległość na podstawie
wykorzystania systemu MOODLE} \\ \hline
			charakterystyka procesowa & właściwości metodyki, zastosowania metod i środków nauki; charakterystyka motywacyjna; organizacyjne formy procesu kształcenia; zarządzanie procesem kształcenia (diagnostyka, projektowanie, regulamin, korekcja); kategoria uczniów, dla których została opracowana technologia & zastosowanie kreatywnych metod nauki \newline– metoda projektów, nauka we współpracy, portfoliów ucznia, metoda problemowa, burza mózgów, forum itd.;\newline środki nauczania – elektroniczne, hipertekstowe, multimedialne materiały, cięgle dostępne na serwerze platformy nauki zdalnej;\newline – motywacja jest osiągniecie celu nauczania przez elastyczne formy, metody, aktualne i ujmujące środki i zasoby, różnorodne formy kontaktu i współdziałania;\newline – organizacyjne formy procesu kształcenia;\newline – lekcja, seminarium, forum, samodzielna praca z zasobami, tekstami, forum, dziennikiem itd.;\newline – zarządzanie procesem nauczania odbywa się przez różne dostępne dla wykładowcy narzędzia: \newline projektowanie – scheduler, struktura kursów, format kursów; \newline diagnostyka – oceny, wiadomości, forum, dialog, automatyczne kopie e-mail, możliwość eksportu ocen do Excela; \newline administrowanie – zbieranie oraz przechowywanie danych o obecności, aktywności, oceny, scheduler, logi, analiza logów, RSS, tworzenie i zarządzanie grupami, poziom praw dostępu; \newline ewaluacja – kwestionariusz, ankieta, głosowanie, logi, dziennik itd. Technologia przewiduje udział w nauce różnych kategorii użytkowników w zależności od celów kształcenia\\ \hline
		\end{tabular}
		\end{table}
		\begin{table}[!h]
		\centering
		\caption[c.d. Analiza technologii pedagogicznej w Moodle'u]{Analiza technologii pedagogicznej kształcenia na odległość na podstawie wykorzystania systemu MOODLE}
		\label{tab:cd1_pedagogika}
		\begin{tabular}{|p{4cm}|p{4cm}|p{9cm}|} \hline
			\textbf{Charakterystyka technologii pedagogicznej} & \textbf{Opis} & \textbf{Opis technologii pedagogicznej
kształcenia na odległość na podstawie
wykorzystania systemu MOODLE} \\ \hline
			zabezpieczenie programowo-metodyczne & plany i programy nauczania;\newline podręczniki i poradniki metodyczne;\newline materiały dydaktyczne;\newline poglądowe techniczne środki nauczania;\newline instrumentarium diagnostyczne & plany i programy nauczania zdalnych
kursów są opracowane przez nauczycieli wykładowców w zależności od przedmiotowego obszaru, wieku uczniów, ich poziomu przygotowania, celów nauki, charakteru szkolnego materiału itd. Plany i programy nauczania zdalnych kursów, jak również ich szczegółowy opis i komentarz, publikuje się na serwerze;\newline materiały dydaktyczne i poradniki metodyczne – w formie elektronicznej, hipertekstowej, multimedialnej; \newline odwołanie się do zasobów Internetu, słowniki tematyczne, cały czas dostępne na serwerze z systemem MOODLE (lekcje, słowniki, zasoby, pliki, linki, katalogi, wiki itd.) \\ \hline
			kryteria oceny technologii pedagogicznej & efektywność; \newline skuteczność & analiza i ocena efektywności i skuteczności
w postaci aktywności, posterów w uczeniu się uczniów (kontrola bieżąca i końcowa), samo i wzajemnej oceny przez uczniów swoich osiągnięcie nauczania odbywa się za pomocą różnorodnych narzędzi, dostępnych w systemie MOODLE:\newline
obecność, aktywność, oceny, Scheduler, logi, analiza logów, RSS, grupy, poziom praw dostępu, komentarze, dziennik, testy, Hot Potatoes Quiz, zadania, lekcja, obecność, aktywność, logi, słownik, seminaria, oceny, obrona projektów (indywidualnych i grupowych), wiadomości, omówienie na forum, możliwość eksportu ocen do Excela, kwestionariusz, ankieta, badanie (głosowanie) itd. \\ \hline
		\end{tabular}
		\end{table}
\textbf{Technologia indywidualizacji nauczania} źródło:\cite{pedagogika}
Indywidualizacja nauczania – forma, model organizacji procesu nauczania-uczenia się, przy którym: \\
	\begin{itemize}
		\item nauczyciel współdziała tylko z jednym uczniem;
		\item jeden uczeń współdziała tylko ze środkami nauczania.
	\end{itemize}
Indywidualne podejscie to:\\
	\begin{itemize}
		\item zasada pedagogiki, zgodnie z która w procesie pracy dydaktyczno-wychowawczej z grupa nauczyciel współdziała z poszczególnymi uczniami według indywidualnego modelu, uwzględniając ich szczególne cechy;
		\item orientacja na indywidualne właściwości uczącego się w kontakcie z nim;
		\item uwzględnienie indywidualnych właściwości uczącego się w procesie nauczania-uczenia się.
	\end{itemize}
Zalety indywidualnego nauczania w tradycyjnej nauce i w kształceniu na odległość z użyciem Moodle przedstawia tabela \ref{tab:indywidualnego} \ref{tab:cd_indywidualnego}, źródło:\cite{pedagogika}
		\begin{table}[!h]
		\centering
		\caption[Indywidualizacja nauczania]{Indywidualizacja nauczania-uczenia się w systemie tradycyjnym i systemie kształcenia na odległość z wykorzystaniem systemu MOODLE}
		\label{tab:indywidualnego}
		\begin{tabular}{|p{4cm}|p{5.5cm}|p{7.5cm}|} \hline
			\textbf{zalety nauczania indywidualnego} & \textbf{warunki realizacji indywidualnego podejścia w nauczaniu tradycyjnym} & \textbf{indywidualizacja nauczania w kształceniu na odległość z wykorzystaniem systemu CLMS MOODLE} \\ \hline
		pozwala całkowicie adaptować zawartość, metody i tempo naukowej działalności uczącego się do jego poziomu i możliwości & proces pracochłonny i trudny do zrealizowania w systemie tradycyjnym: klasowo-lekcyjnoprzedmiotowym & w systemie MOODLE zawartość, metody i tempo szkolnej działalności uczącego się można dostatecznie łatwo, szybko i efektywnie adaptować do jego poziomu i możliwości, dzięki giętkiemu systemowi ustawień i parametrów; można zmienić format kursu, strukturę, jego zawartość i stosowane metody \\ \hline
		pozwala śledzić każde działanie uczącego się i wykonywane operacje przy rozwiązywaniu konkretnych zadań & proces trudny, prawie niewykonalny dla realizacji w systemie tradycyjnym klasowo-lekcyjnym z powodu braku niezbędnych mechanizmów i narzędzi, jak również z powodu wciąż dużej liczby uczniów w jednej klasie & w systemie MOODLE dostępne jest całe spektrum pożytecznych i efektywnych mechanizmów, pozwalających śledzić aktywność ucznia, jego osiągnięcia, realizacje tych lub innych zadań i odpowiedzi na testy, urzeczywistniać pełny monitoring pracy z danym kursem nauczania, co może pomóc w zbudowaniu i korekcji indywidualnej drogi nauczania danego uczącego się \\ \hline
		pozwala śledzić jego osiągnięcia od braku wiedzy do wiedzy & proces śledzenia osiągnięć ucznia od braku wiedzy do wiedzy ma ograniczone możliwości & śledzenie osiągnięć ucznia w systemie MOODLE jest permanentne i wszechstronne, rezultaty mogą być przedstawione w różny sposób: w postaci tabeli, wykresu, w punktach, w procentach na tle poprzednich rezultatów danego ucznia albo na tle klasy \\ \hline
		\end{tabular}
		\end{table}

		\begin{table}[!h]
		\centering
		\caption[c.d. Indywidualizacja nauczania ]{Indywidualizacja nauczania-uczenia się w systemie tradycyjnym i systemie kształcenia na odległość z wykorzystaniem systemu MOODLE}
		\label{tab:cd_indywidualnego}
		\begin{tabular}{|p{4cm}|p{5.5cm}|p{7.5cm}|} \hline
			\textbf{zalety nauczania indywidualnego} & \textbf{warunki realizacji indywidualnego podejścia w nauczaniu tradycyjnym} & \textbf{indywidualizacja nauczania w kształceniu na odległość z wykorzystaniem systemu CLMS MOODLE} \\ \hline
			pozwala wnosić w porę niezbędne korekcje do działalności, zarówno uczącego się, jak i nauczyciela & procedura jest bardzo utrudniona z powodu braku sprzężenia zwrotnego i współdziałania miedzy uczącym się a nauczycielem & dzięki giętkiej i elastycznej strukturze modułowej i koncepcji funkcjonowania, system MOODLE, opiera się na zasadach konstruktywizmu, uwzględniając i realizując idee pedagogiki kognitywnej i częściowo nauczania programowanego; pozwala w każdej chwili wnosić niezbędne korekty do kursu, jak również do działalności zarówno uczącego się, jak i nauczyciela; \\ \hline
		pozwala przysposabiać ich do wciąż zamieniającej się, lecz kontrolowanej sytuacji ze strony nauczyciela oraz ucznia & ten proces powinien był być ściśle związany ze stałym monitoringiem procesu nauczania uczenia się i jego rezultatów, ich analizy, które w nauczaniu tradycyjnym osiągnięte praktycznie być nie może & w systemie istnieje obiektywny permanentny monitoring wszystkich odbywających się procesów, tak na poziomie ucznia(ów), jak i na poziomie nauczyciela(i): logi, aktywność, długotrwałość sesji, rezultaty nauczania, komentarze ze strony nauczyciela, kolegów, wykorzystanie tych lub innych zasobów, narzędzi dla kontroli (samokontroli), prośba o pomoc do nauczyciela lub kolegów w grupie itd.; w zależności od otrzymanych danych nauczyciel w każdej chwili może wnieść zmiany w proces nauczania-uczenia się \\ \hline
		\end{tabular}
		\end{table}
