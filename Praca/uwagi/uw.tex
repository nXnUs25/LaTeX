\hspace{1cm} Przyglądając się stronie technicznej Moodle'a widać że jego struktura jest taka sama jak ogólna struktura platformy edukacyjnej. Więc aby podsumować i ująć wszystko w kilku słowach możemy powiedzieć, że potrzebny nam serwer, inaczej komputer, który działa przez 24 godziny, jest wpięty do sieci Internet i ma rozpoznawalny adres IP. Następnie musimy mieć zainstalowaną platformę Moodle'a. Cel ten udało się osiągnąć. W tym przypadku skorzystałem z usług hostingowych ze strony \href{http://fazz.pl/}{\textit{http://fazz.pl/}}. Jest to jeden z darmowych hostingów. Na jednym z darmowych kont można śmiało postawić Moodle'a. Proces i plan tworzenia kursu został omówiony w rozdziale wyżej. W projekcie widać, utworzenie prostego kursu nie stanowi żadnego problemu nawet dla laika. Dodawanie materiałów do kursu również nie stanowi większych problemów. Dzięki dobremu interfejsowi platforma Moodle'a jest platformą bardzo intuicyjną. I można samemu idąc drogą eksperymentów stworzyć dobry kurs czy też dobrze przygotowany kawałek materiałów. Mój kurs miał za zadanie przeprowadzić użytkownika przez proces instalacji co udało się zrealizować. \\
\ \\
Platforma Moodle'a jest intuicyjna w obsłudze i została zaopatrzona w przydatny system pomocy, dostępny na oficjalnej stronie \href{http://moodle.org}{\textit{moodle.org}}. Użytkownik znajdzie tam dokładne informacje dotyczące korzystania z każdej funkcji Moodle'a. Do zalet systemu Moodle zalicza się różnorodność zasobów edukacyjnych:
\begin{itemize}
	\item Teksty pisane
	\item Teksty mówione
	\item Animacje
	\item Prezentacje
	\item Rebusy
	\item Hiperłącza
\end{itemize}
Struktura Moodle'a odpowiada potrzebą szkół i uczelni, co widać w sieci. Prawie każda uczelnia posiadająca swoja platformę edukacyjną bazuje ją na Moodle'u. Chociażby nasza politechnika. Która również dla swoich potrzeb wdrożyła dany system. Tworząc kurs korzystamy z pewnych form stworzonych specjalnie pod to zadanie, ale każdy autor kursu decyduje sam o jego specyfice. Autor kursu ma możliwość wprowadzania korekt do działalności, zarówno uczącego się jak i nauczyciela. Tworząc kurs posiadamy różne możliwości zarządzania kursem poprzez takie narzędzia jak:
\begin{itemize}
	\item Lekcje
	\item Warsztaty
	\item Głosowania
	\item Ankiety
	\item Zadania
	\item Fora
	\item Czaty
	\item Opisy
	\item Testy
	\item Dzienniki
\end{itemize}
Na mojej witrynie do zarządzania kursem zostały użyte takie narzędzia jak czaty i fora. Podczas prowadzenia kursów mamy możliwości dodawania dodatkowych materiałów które rozszerzają naszą wiedzę w danym temacie. W kursie mamy również możliwość wielokrotnego powracania do konkretnych treści. Kurs nie przewiduje dzielenia na grupy, które są zbyteczne przy takiej formie kursu, ale istnieje możliwości dzielenia uczniów na grupy. Jest to pomocne w przypadku chęci podzielenia uczniów ze względu na klasy. Witryna posiada również kalendarz który pozwala przypominać użytkownikom o różnych zdarzeniach. Witryna pozwala na śledzenie każdego działania użytkownika, pokazuje operacje jakie wykonał użytkownik przy rozwiązywaniu konkretnych zadań. Mamy również możliwość adaptować zawartość, metody i tempo uczenia się, ze względu na poziom i możliwości uczącego się. Więc można stwierdzić z całą pewnością, że postawiony cel którym było stworzenie przykładowej witryny e-learning-owej wspomagającej zdalne nauczanie został osiągnięty przy pomocy systemu Moodle'a. Pokazuje to, że Moodle stanowi dużą konkurencje i alternatywę dla komercyjnych platform e-learning-owych. Dzięki budowie modułowej jesteśmy wstanie w łatwy sposób tworzyć kursy i dodawać do nich nowe treści. Moodle obserwuje bardzo duży i szybki wzrost użytkowników. I co za tym idzie zwiększa się jego społeczność, gdzie to właśnie ona jest odpowiedzialna za rozwój tej platformy.\\
\ \\
Praca można potraktować jako plan projektu, który można zrealizować. Poszczególne rozdziały dostarczają wiedzy na temat e-learningu oraz jego idei. Pomagają również w podjęciu decyzji zgodnych z postawionym wcześniej celem, które dany użytkownik chce osiągnąć tworząc własną witrynę nauczania. Podczas tworzenia witryny wykonuje się zazwyczaj określona sekwencję kroków, które zostały zaprezentowane w pracy. Poprzez własne eksperymenty można w łatwy sposób wzbogacić swoją witrynę poprzez zwiększenie jej możliwości. Moodle jest platformą która dzięki przyjaznemu interfejsowi zachęca do eksploracji zawartych w systemie funkcji. W pracy dużą część poświęcono na pokazaniu zalet jakie niesie za sobą korzystanie z e-learningu. \\
\ \\
Podczas pisania pracy nie zabrakło i problemów. Jednym z pierwszych problemów było znalezienie darmowego hostingu. Gdyż nie wszystkie darmowe konta spełniają możliwości platformy Moodle'a. Problem ten po większej analizie usług hostingowych został rozwiązany. Następnym krokiem był proces instalacji i konfiguracji platformy, gdzie obyło się bez większych niespodzianek. Proces instalacji w sam w sobie jest zadaniem prostym i nawet ktoś z nie wielką wiedzą informatyczną jest w stanie zainstalować tego typu platformę. Projekt witryny został stworzony jakiś czas temu. Podczas upływającego czasu usługodawca hostingu zmienił serwery. I co za tym idzie zmieniła się darmowa domena dla platformy. Co w znacznym stopniu utrudniło przywracanie backup'u. Po przywróceniu całej platformy należało zmienić ręcznie ścieżki do obrazków czy też innych źródeł zawartych w materiałach. Gdzie bazy danych również musiały być przywracane ręcznie przy pomocy panelu phpMyAdmin i skryptów SQL. Większość prac nad witryną przebiegała sprawnie i nie wymagała dużej ingerencji z mojej strony w kod źródłowy. \\
\  \\
Witrynę możemy rozwijać wedle jakich kol wiek upodobań. Moodle jest systemem z bardzo dużą opcją możliwości. \\
Zacznijmy od sposobów uwierzytelniania, dostępne opcje to:
\begin{itemize}
	\item Uwierzytelnienie z wykorzystaniem poczty elektronicznej (opcja ta jest wykorzystana w projekcie)
	\item Użycie serwera CAS\footnote{CAS~-~Centralny System Uwierzytelniania (ang. Central Authentication Service) CAS to specjalny program. W dużym uproszczeniu służy do pilnowania, gdzie komu w konkretnym systemie informatycznym (lub ich zbiorze) wolno zajrzeć. Sam CAS nie przechowuje jednak danych użytkowników, takich jak hasła. Musi współpracować z jakimś zbiorem informacji o użytkownikach.}.
	\item Zewnętrzna baza danych
	\item Użycie serwera IMAP\footnote{IMAP~-~(ang. (Internet Message Access Protocol) jest protokołem warstwy aplikacji w architekturze protokołów Internetowych. Jego głównym zadaniem jest umożliwienie stacji roboczej dostępu do listów elektronicznych znajdujących się w skrzynce pocztowej na serwerze pocztowym. }
	\item Użycie serwera LDAP\footnote{LDAP~-~(ang. Lightweight Access Directory Protocol) serwer usług katalogowych.}.
	\item Moodle Network authentication.
	\item Użycie serwera NNTP\footnote{NNTP~-~(ang. Network News Tranport Protocol), jak nazwa wskazuje jest protokołem "na którym" działają niusy.}
	\item Brak uwierzytelniania.
	\item PAM\footnote{System uwierzytelniania w systemie Linux wykorzystuje mechanizm PAM (Pluggable Authentication Modules - Dołączalne Wtyczki Uwierzytelniające). Implementację dla systemu Linux stworzył i cały czas rozwija Andrew G.Morgan. System PAM to zestaw bibliotek i wtyczek, które są wykorzystywane do uwierzytelniania użytkowników w systemie. Biblioteka jest wykorzystywana przez aplikację, aby wywołać procedurę uwierzytelniającą. Wtyczki określają możliwości systemu uwierzytelniającego, możliwościami taki są ograniczenie czasu logowania do konkretnych godzin, określenie różnych źródeł danych o użytkownikach (na przykład baza LDAP, MySQL i inne). }.
	\item Użycie serwera POP3\footnote{POP3~-~Post Office Protocol version 3 to protokół internetowy z warstwy aplikacji pozwalający na odbiór poczty elektronicznej ze zdalnego serwera do lokalnego komputera poprzez połączenie TCP/IP. Ogromna większość współczesnych internautów korzysta z POP3 do odbioru poczty.}.
	\item Użycie serwera RADIUS\footnote{RADIUS~-~(ang. Remote Authentication Dial In User Service), usługa zdalnego uwierzytelniania użytkowników. Protokół używany do uwierzytelniania. Stosowany przez dostawców internetowych na serwerach innych niż serwery z systemem Windows. Obecnie jest najpopularniejszym protokołem podczas uwierzytelniania i autoryzowania użytkowników sieci telefonicznych i tunelowych.}.
	\item Shibboleth\footnote{Shibboleth~-~działa jako servlet, sam nie przechowuje danych, pobiera je z innych źródeł np. LDAP}
\end{itemize}
Następnym rozwinięciem czy też zmianą naszego projektu jest zmiana wyglądu. Wygląd witryny możemy zmieniać poprzez stworzenie własnego design, lub też skorzystanie z już dostępnych kompozycji. Aby zmienić wygląd należy skorzystać z bloku \textbf{Administracja serwisu}, a następnie 
\begin{verbatim}
	Wygląd->Tematy->Wybór kompozycji.
\end{verbatim}
Do witryny możemy dodawać dowolnej treści kursy. Poprzez tworzenie większej ilości kursów nasza witryna będzie się rozwijała i powiększała swoją liczbę użytkowników. \\
Podsumowując, witrynę zbudowana na podstawie systemu Moodle'a daje nam bardzo dużą możliwość działania. Wygląd i ustawienia witryny głównie zależą tylko i wyłącznie od administratora. On decyduje jakie procesy i jakie kursy będziemy mogli włączać do naszej platformy aby zwiększyć atrakcyjność naszej witryny.\\
\\
Tworząc witrynę z kursem pokazującym jak zainstalować i skonfigurować Moodle'a, pierwszym etapem jaki wykonałem po zainstalowaniu Moodle'a, była jego konfiguracja. W początkowej fazie konfiguracji zająłem się sposobem rejestracji użytkownika do systemu. Wybrałem funkcje uwierzytelniania z wykorzystaniem poczty elektronicznej plus korzystanie z serwisu \textit{reCAPTCHA}. Aby system współdziałał z \textit{reCAPTCHA} należy na stronie \href{http://recaptcha.net}{\textit{http://recaptcha.net}} wygenerować dla danej strony klucz publiczny i klucz prywatny, a następnie przejść do \textit{Administracja serwisu/Użytkownicy/Uwierzytelniania/Zarządzaj uwierzytelnianiem} i w odpowiednich polach podać wygenerowane klucze. Również dodatkową opcją z której skorzystałem tworząc witrynę było skonfigurowanie współpracy z narzędziem \textit{GeoIP}. Dodanie tej usługi zacząłem od utworzenia w folderze \textit{moodledata} katalogu \textit{geoip} i podaniu ścieżki dostępu do pliku. W moim przypadku ścieżka dostępu wyglądała tak \textit{/home/nonus25/domains/moodledata/geoip/GeoLiteCity.dat}. Gdzie należy umieścić uprzednio pobrany plik \textit{GeoLiteCity.dat} z adresu\\ \href{http://www.maxmind.com/download/geoip/database/GeoLiteCity.dat.gz}{\textit{http://www.maxmind.com/download/geoip/database/GeoLiteCity.dat.gz}}. Następnie przeszedłem do strony \href{http://code.google.com/apis/maps/signup.html}{\textit{http://code.google.com/apis/maps/signup.html}} gdzie wygenerowałem API klucz dla \textit{Google Maps} podając swoją domenę \href{http://moodle.chmielua.fazz.pl}{\textit{http://moodle.chmielua.fazz.pl}}. Wygenerowany klucz należy wpisać w odpowiednie pole znajdujące się na stronie\\ \textit{Administracja serwisu/Lokalizacja/Ustawienia lokalizacji}. Więcej na temat moich prac jakie wykonywałem tworząc witrynę znajduje się w rozdziale \textit{Projekt systemu~-~Konfiguracja witryny}. Kompozycja wyglądu platformy została pobrana przez zemnie ze strony \\ \href{http://moodle.org/mod/data/view.php?d=26}{\textit{http://moodle.org/mod/data/view.php?d=26}}. Pobrany pakiet należy umieścić w katalogu \textit{themes}. W pobranej kompozycji dokonałem kilku zmian zamieniając główną ikonę platformy (jest to ikona znajdująca się w katalogu \textit{themes/custom\_corners/favicon.ico}), oraz dokonałem zmiany domyślnego obrazka dla użytkownika. W kodzie źródłowym dokonałem małych zmian co do wyglądu nagłówka jak i stopki strony. Są to pliki znajdujące się \textit{themes/custom\_corners/}. Za nagłówek strony odpowiedzialny jest plik \textit{header.html} gdzie dodałem swój kawałek kodu 
\begin{verbatim}
<img  class="headermain" 
src="http://moodle.chmielua.fazz.pl/images/moodle/eLearningLogoTag-sm.png" 
alt="E-learning moodle.chmielua.fazz.pl" 
/>
\end{verbatim}
i za komentowałem kod odpowiadający za wyświetlanie napisu w tym miejscu 
\begin{verbatim}
<!-- <h3 class="headermain" color><?php echo $heading ?></h3> -->
\end{verbatim}
w dwóch lokalizacjach:
 \begin{verbatim} <?php print_container_start(true, '', 'header-home'); ?>
<?php print_container_end(); ?>\end{verbatim}
oraz
\begin{verbatim}<?php print_container_start(true, '', 'header'); ?>
<?php print_container_end(); ?>\end{verbatim}
Za stopkę witryny odpowiedzialny jest plik \textit{footer.html} gdzie moje zmiany wyglądały w następujący sposób:
\begin{verbatim}
    print_container_start(false, '', 'footer');
    echo '<p class="helplink">';
    //echo page_doc_link(get_string('moodledocslink'));
    echo '<a href="http://chmielua.blogspot.com/search/label/Moodle">
	Trochę inaczej o Moodle</a>';
    echo '</p>';
   
    //echo $loggedinas;
    echo '<img src="http://moodle.chmielua.fazz.pl/
		images/moodle/elearning_logo.png" />';
    //echo $homelink;\end{verbatim}
Dokonałem również zmiany układu strony, aby zmienić układ kolumn na stronie należy dodać następującą kod w pliku \textit{config.php}, 
\begin{verbatim}
$THEME->layouttable = array('middle', 'right', 'left');
\end{verbatim}
