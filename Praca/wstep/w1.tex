Aby zachęcać uczących do korzystania z zdalnego sposobu nauczania jest ważnym aby zagwarantować wysoki poziom nauczania. Aby uzyskać odpowiednią jakość takiej edukacji należy nie uwzględniać tradycyjnej metodyki na jakich opiera się kształcenie stacjonarne. \\
\ \\
Sposoby zdalnego nauczania posiadają swoje zalety jak i wady. Jedną z głównych zalet e-learningu jest ruchomy czas pracy i wygodę uczących się, w szczególności kiedy mają oni jeszcze inne zobowiązania takie jak np. praca, dom, rodzina itp.. System e-learningu ułatwia również komunikację między uczniami, wzbogaca sposób nauki poprzez wprowadzanie multimediów i nie werbalnej prezentacji materiału. Platformy te pozwalają uczniom uczyć się we własnym tempie, jak również pozwalają nauczycielom na kontrolowanie tępa nauki. Porównując e-learning z tradycyjnymi zajęciami w klasie, e-learning przynosi dużo większe zyski organizującym dane szkolenia czy też kursy. \\
\ \\
Oczywiście nie brakuje krytyki co do sposobu nauczania na odległość. Jedna z najczęściej wymienianych wad takiego systemu nauczania jest brak osobistego kontaktu z nauczycielem. Często również mówi się o wrażeniu odosobnienia przez uczących się, które jest niwelowane poprzez zastosowanie blogów, czatów, forów dyskusyjnych itp.\\
\ \\
Wraz z rozwojem zdalnej edukacji i z chęcią utrzymania wysokiej jakości świadczonych usług poprzez e-learning pojawiły się próby opracowania uniwersalnych kryteriów dobrego zdalnego kursu/szkolenia. Tego typu kryteria powinny być uwzględniane we wszystkich aplikacjach wspomagających zdalne nauczanie. Gdyż mają one na celu zapewnienie wysokiego poziomu nauczania. Zasady te wpływają na sposób organizacji zdalnego kształcenia, na zawartość i postać materiałów dydaktycznych, a także na sposób przekazywania wiedzy. \\
\ \\
Podczas tworzenia kursu/szkolenia należy zwrócić uwagę na: \\
Reguły organizacyjne: \\
	\begin{itemize}
		\item udostępnienie w Internecie opisu kursu/szkolenia,
		\item zapewnienie wstępnego szkolenia w zakresie nawigacji i używania dostępnych funkcji, 
		\item zapewnienie osobie nauczanej możliwości łatwego i szybkiego porozumiewania się zarówno z osobą nauczającą, jak i z innymi uczestnikami kursu/szkolenia, 
		\item umożliwienie wypowiadania się osób nauczanych oraz osoby nauczającej na forum całej "wirtualnej klasy", 
		\item ustalenie terminów, w których cała wirtualna klasa będzie dostępna online. 
	\end{itemize} 
\ \\
Zasady projektowania materiałów dydaktycznych: \\
	\begin{itemize}
		\item materiały dydaktyczne powinny być atrakcyjne, 
		\item materiały dydaktyczne stworzone dla zdalnego nauczania powinny spełniać podobne funkcje, jak materiały tradycyjne (tzn. wykorzystywane w tradycyjnym, stacjonarnym nauczaniu), 
		\item materiały dydaktyczne powinny zawierać odnośniki do innych stron internetowych, związanych z danym materiałem, 
	\end{itemize}
\ \\
Zasady przekazywania wiedzy: \\
	\begin{itemize}
		\item prezentowanie materiałów dydaktycznych w sposób logiczny, zgodny z określoną ścieżką dydaktyczną, przy czym osoby nauczane powinny mieć możliwość pewnych modyfikacji tej ścieżki, 
		\item prezentowanie materiałów dydaktycznych w sposób dostosowany do różnych stylów uczenia się ludzi, 
		\item podtrzymywanie koncentracji osoby nauczanej na prezentowanym materiale, 
		\item używanie poprawnego języka, zrozumiałego dla osoby nauczanej, 
		\item sprawna i szybka prezentacja materiałów dydaktycznych 
	\end{itemize}
\ \\
Ogólne zalecenia: \\ 
	\begin{itemize}
		\item zapewnienie pełnej funkcjonalności prowadzonego kursu/szkolenia, 
		\item zapewnienie kontaktu z niezależnymi ekspertami, którzy swoją wiedzą mogą wesprzeć i uatrakcyjnić proces dydaktyczny, 
		\item zwrócenie szczególnej uwagi na sposoby kontroli wiedzy przyswajanej przez osoby nauczane. 
	\end{itemize}
\ \\
Stosując się do wyżej wymienionych reguł jest łatwiej nam utrzymać nasz kurs/szkolenia na wysokim poziomie. Reguły te niestety mówią tylko nam o tym jak i czym mamy się kierować budując nasz kurs/szkolenie, anie, jak to zrobić.
