\hspace{1cm} Wraz z rozwojem nowych technologii otwierają się nowe ścieżki pozwalające na zmiany w wielu dziedzinach naszego życia, w tym także i sposoby edukacji. W raz z rozpowszechnianiem się internetu pojawiły się nowe formy zdalnego nauczania. Oczywiście osoba chcąca korzystać z takiego sposobu nauczania musi posiadać sprzęt komputerowy, który natomiast powinien być odpowiednio skonfigurowany oraz mieć dostęp do Internetu. Nauczanie za pomocą komputera może się odbywać na różne sposoby. \\
Dwie podstawowe formy jakie przyjmuje \textit{e-learning} to:
\begin{itemize}
	\item CBT \footnote{Computer Based Training} -~szkolenie oparte na technologii komputerowej.
	\item WBT \footnote{Web Based Training} -~szkolenie wykorzystujące sieć globalną.
\end{itemize}

W przypadku drugiego sposobu nauczania można użyć także pojęcia Online Learning, znaczącego nic innego jak nauczanie "na żywo" z wykorzystaniem sieci komputerowej. Tryb ten nazywamy trybem synchronicznym. Zaś pierwszy typ szkoleń nazywany jest trybem asynchronicznym. \\
\ \\
Do pierwszego typu szkoleń możemy zaliczyć wszelkie kursy multimedialne. Bazują one na różnych nośnikach danych, takich jak CD-ROM, DVD, Pen Drive i wszelkie inne media, które można z powodzeniem wykorzystywać w pracy na komputerze. \\
\ \\
Drugi typ szkolenia to szkolenia wykorzystujące technologie komputerowe i sieci rozległe, również Internet. Internetowe platformy e-learning-owe mogą być wykorzystywane do prowadzenia nie zależnych szkoleń, ale także mogą być elementem wspomagającym lub uzupełniającym tradycyjne formy szkoleń. Nawet w pewnych przypadkach mogą stanowić alternatywę dla niektórych tradycyjnych kursów. W tym przypadku tylko uczący decydują którą formę wybiorą. \\






