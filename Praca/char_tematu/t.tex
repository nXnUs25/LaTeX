\hspace{1cm} Przy wyborze tematu kierowałem się rosnącym popytem na tego typu usługę. Coraz większa liczba osób uczących się decyduje się na korzystanie z tej formy edukacji. W chwili obecnej istnieje wiele serwisów e-learning-owych. Moją uwagę zwróciła platforma Moodle'a. Moodle jest darmowym systemem zarządzania nauczaniem. Pozwala ona użytkownikom w pełni korzystać z przyjemności jaką daje nauka online, która w dalszym ciągu poszerza swoje horyzonty. Każdy kto planuje utworzyć witrynę nauczania powinien rozważyć użycie tejże platformy. Moodle w sam sobie został zaprojektowany tak aby wspierać sposoby uczenia na odległość. Z stąd zainteresowanie platformą rośnie szybko. Wraz ze wzrostem popularności Moodle'a zwiększa się zapotrzebowanie na opcje pozwalające na wymuszenie liniowego przebiegu kursów. Cały czas rozwijane są moduły przeznaczone do otwierania i zamykania kursów lub ich elementów na podstawie wyników uczniów w poprzednich kursach czy lekcjach. Aby poznać aktualny etap rozwoju tych modułów, można przejrzeć dział z najnowszymi informacjami oraz strony modułów na oficjalnej witrynie pod adresem \href{http://www.moodle.org}{http://www.moodle.org}. \\
\ \\
Realizując temat swojej pracy chciałbym przede wszystkim skupić się na pokazaniu użytkownikom platformy, że każdy jest w stanie stworzyć swój własny kurs online. Że nie potrzeba być programistą aby zaprojektować i udostępnić swój kurs online. \\
\ \\
Korzystając z danych zawartych na stronach Wikipedii przedstawię tutaj kilka danych statystycznych: \\
\ \\
Spośród pierwszych organizacji które wykorzystywały e-learning w latach 80. wymienić można: Zachodni Instytut Psychologii Behawioralnej (z ang. Western Behavioral Sciences Institute), Instytut Technologii w Nowym Jorku ( z ang. New York Institute of Technology), Elektroniczny System Wymiany Informacji (z ang. Electronic Information Exchange System - EIES), Instytut Technologii w New Jersey (z ang. New Jersey Institute of Technology) oraz Zintegrowana Edukacja (z ang. Connected Education). W późniejszych latach również organizacja Niezależne Media Studenckie (z ang. Independent Student Media) opracowała roboczy program nauczania dla studentów realizowany za pomocą interaktywnego podręcznika online (z ang Interactive Online Textbook). \\
\ \\
Według raportu opracowanego przez Konsorcjum Sloan (z ang. Sloan Consortium), wiarygodne źródło informacji na temat szkolnictwa wyższego, do 2003 roku liczba studentów korzystających z platform e-learning-owych w Stanach Zjednoczonych wyniosła ponad 1,9 mln. \\
\ \\
Zaskakujący wzrost liczby użytkowników wynoszący obecnie około 25 procent w skali roku poważnie zmienił wcześniejsze statystyki. \\
\ \\
Konsorcjum Sloan podaje, iż obecnie, praktycznie wszystkie państwowe instytucje szkolnictwa wyższego jak i przeważająca większość odpłatnych szkół wyższych oferuje zajęcia online. Dla porównania tego typu zajęcia są prowadzone zaledwie w połowie nieodpłatnych uczelni prywatnych. Raport Sloana opracowany na podstawie sondażu przeprowadzonego na najlepszych wyższych uczelniach dowodzi, że studenci są przynajmniej tak zadowoleni z zajęć online jak z kursów tradycyjnych. W miarę obniżania się kosztu wprowadzenia takiego systemu uczelnie prywatne mogą bardziej zaangażować się w prezentacje online Do pracy online ze studentami należy zatrudnić odpowiednio wyszkoloną kadrę, której członkowie muszą posiadać nie tylko odpowiednią wiedzę merytoryczną , ale też wysokie kwalifikacje w obsłudze komputera i internetu. \\
\ \\
Popularna stała się również koncepcja tzw. Digital Native (osoba mająca styczność z technologią od najmłodszych lat). Z pewnością na przyszłość e-learningu wpływ będą miały różnice pokoleniowe, jednak w miarę wzrostu liczby dorosłych studentów będą one zanikać. \cite{wiki_e-l} \\
\ \\
Na koniec tego rozdziału chciałbym polecić pięć top Polskich blogów poświęconych tematyce e-learningu, oto one:
	\begin{itemize}
		\item \href{http://testabz.ning.com/}{http://testabz.ning.com/}
		\item \href{http://edukacjaprzyszlosci.blogspot.com/}{http://edukacjaprzyszlosci.blogspot.com/}
		\item \href{http://www.enauczanie.com/ }{http://www.enauczanie.com/ }
		\item \href{http://e-learning.blog.pl/}{http://e-learning.blog.pl/}
		\item \href{http://rapid-elearning.blogspot.com/}{http://rapid-elearning.blogspot.com/}
	\end{itemize}
