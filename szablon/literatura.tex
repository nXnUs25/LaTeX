%Sprawdzi� poprawno�� i kolejno�� element�w w literaturze

\begin{thebibliography}{100}

	\bibitem{Bandemer} Bandemer H., Gottwald S., {\em	Fuzzy Sets, Fuzzy Logic, Fuzzy Methods with Applications}, 	
		John Willey and Sons, England, 1995.

	\bibitem{UML} Booch G., Rumbaugh J., Jacobson I., {\em UML prze\-wo\-dnik u\-�yt\-kow\-ni\-ka}, Wydawnictwo 
		Nau\-ko\-wo\--Tech\-ni\-czne, Warszawa.

	\bibitem{Buckley} Buckley J.J., Eslami E., {\em An Introduction to Fuzzy Logic and Fuzzy 
		Sets}, Physica-Verlag, Heidedlberg, 2002.
		
	\bibitem{KacprzykZadrozny} Kacprzyk J., Zadrozny S., {\em On Linguistic Approaches in Flexible Querying and Mining of Association Rules}, In. Larsen H.L., Kacprzyk J., Zadro�ny S., Andreasen T., Christiansen H., {\em Flexible Query Answering Systems}, Physica Varlag, Heidelberg, 475-484, 2001.
	\bibitem{KacprzykZadrozny1} Kacprzyk J., Zadrozny S., {\em Computing with words in intelligent database querying:standalone and Internet-based applications}, Information Sciences, 34, 71-109, 2001.
	\bibitem{KacprzykZadrozny2} Kacprzyk J., Zadro�ny S., {\em Fuzzy querying for Microsoft Access}, In. Proceedings of the Third IEEE Conference on Fuzzy Systems, Orlando, USA, col. 1, 167-171, 1999.
	\bibitem{KacprzykZadrozny3} Kacprzyk J., Zadro�ny S., {\em Fuzzy queries in Mocrosoft Access: towards a 'more intelligent' use of Microsoft Windows based DBMSs}, In. Proceedings of the Second Australian and New Zealand Conference on Intelligent Information Systems-ANZIIS'94, Brisbane, Australia, 492-496, 1994.
	\bibitem{KacprzykZadrozny4} Kacprzyk J., Zadro�ny S., {\em FQUERY for Access: fuzzy querying for a Windows-based DBMS}, In. Bosc P., Kacprzyk J., {\em Fuzziness in Database Management System}, Physica-Verlag, Heidelberg, 415-433, 1995.
	\bibitem{KacprzykZadrozny5} Kacprzyk J., Zadro�ny S., {\em Flexible querying using fuzzy logic: An implementation for Microsoft Access}, In. Andreasen T., Christiansen H., Larsen H.L., {\em Flexible Query Answering System}, Kluwer Academic Publishers, Boston, 247-275, 1997.
	\bibitem{KacprzykZadrozny6} Kacprzyk J., Zadro�ny S., {\em Implementation of OWA operators in fuzzy querying for Microsoft Access}, In. Yager R.R., Kacprzyk J., {\em The Ordered Weighted Averaging Operators: Theory and Applications}, Kluwer Academic Publishers, Boston, 293-306, 1997.

	\bibitem{Leski} ��ski J., Straszecka E., {\em	Zbiory rozmyte i ich zastosowanie w diagnostyce medycznej},	In. 
		Zajdel R., K�cki E., Szczepaniak P., Kurzy�ski (red.), {\em Kompendium informatyki medycznej}, 
		Medica-press, Bielsko-Bia�a, 2003.

	\bibitem{Ochelska2001} Ochelska J., Niewiadomski A., Szczepaniak P. S., {\em Linguistic Summaries 
		Applied To Medical Textual Databases}, Dept. of Electronics \& Computer Systems University 	
		of Silesia, Ustro�, 2001.
	\bibitem{Ochelska2004} Ochelska J., Szczepaniak P., Niewiadomski A.,	{\em Automatic Summarization on 
		Standarized Textual Databases Interpreted in Terms of Intuitionistic Fuzzy Sets}, W: Soft Computing Tools, 
		Techniques and Applications, Akademicka Oficyna Wydawnicza EXIT, Warszawa, 2004.
	\bibitem{Ochelska2005} Ochelska J., Szczepaniak P.S., {\em Textual Fuzzy Similarity and Sequence Kernels}, XIII KOnferencja Sieci i Systemy Informatyczne, ��d�, 299-304, 2005.

	\bibitem{Rasiowa} Rasiowa H.,	{\em Wst�p do matematyki wsp�czesnej}, 
		Wydawnictwo Naukowe PWN, Warszawa, 1998.
		
	\bibitem{Rutkowscy} Rutkowska D., Pili�ski M., Rutkowski L., {\em Sieci neuronowe, 
		algorytmy genetyczne i systemy rozmyte}, Wydawnictwo Naukowe PWN, Warszawa-��d�, 1997.

	\bibitem{YagerPodsumowania1} Yager R. R., T. C. Rubinson, {\em Linguistic Summaries of Data 
		Bases}, Proc. IEEE Conference on Decision and Control, San Diego, 1981.
	\bibitem{YagerPodsumowania2} Yager R. R., KFord. M., Canas A. J, {\em On Linguistic Summaries 
		of Data, w Information Processing and Management of Uncertainty in Knowledge-Based System}, 
		3rd International Conference, Paris, France, 1990. 
	\bibitem{YagerPodsumowania4} Yager R. R., {\em On Linguistic Summaries of Data}, w Knowledge 
	  Discovery in Databases, Piatetsky-Shapiro, G. \& Frawley, B. (eds.), Cambridge, 1991.
	\bibitem{YagerPodsumowania3}	Yager R. R., {\em Linguistic summaries as a tool for database 
		discovery}, Workshop on Fuzzy Database System and Information Retrival, Yokohama, 
		Japan 1995. 

	\bibitem{Zadeh} Zadeh L.A., {\em Fuzzy Sets}, Information and Control 8, 1965.
\end{thebibliography}